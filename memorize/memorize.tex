\documentclass{article}

\usepackage{amssymb, amsmath, amsfonts}
\usepackage{mathtools}
\usepackage{physics}
\usepackage{enumerate}
\usepackage{array}
\usepackage[margin=0.6in]{geometry}

\newcommand{\E}{\varepsilon}
\newcommand{\ran}{\mathrm{ran}\,}
\newcommand{\ind}{\mathrm{ind}\,}
\newcommand{\Lip}{\mathrm{Lip}\,}
\newcommand{\supp}{\mathrm{supp}\,}
\newcommand{\sgn}[1]{\mathrm{sgn}\left[#1\right]}
\newcommand{\f}[3]{#1\ :\ #2 \rightarrow #3}
\def\Rl{\mathbb{R}}
\def\Cx{\mathbb{C}}
\def\hilb{\mathcal{H}}
\def\torus{\mathbb{T}}
\def\topo{\mathcal{T}}


\begin{document}

\section{Sobolev Embeddings}
    \subsection{On $\torus$}
        Let $k > \frac{1}{2}$.  Then
        \begin{align*}
            \norm{f}_{C(\torus)} \leq C\norm{f}_{H^k(\torus)}
        \end{align*}
        That is, if $f$ has at least half of a derivative on a $1$-dimensional compact subset of $\Rl$, then it is continuous.
    \subsection{On compact subsets of $\Rl^2$}
        Let $\Omega \Subset \Rl^2$ and $kp > 2$.  Then
        \begin{align*}
            \norm{f}_{L^\infty(\Omega)} \leq C\norm{f}_{W^{k,p}(\Omega)}
        \end{align*}
        That is, functions on compact intervals with enough derivatives, and/or are $p$-integrable enough, are bounded.
    \subsection{On $\Rl^n$, with $k = 1$}
        Let $p > n$.  Then
        \begin{align*}
            \norm{f}_{C^0(\Rl^n)} \leq \underbrace{\norm{f}_{C^{0,1 - \frac{n}{p}}(\Rl^n)}}_{\text{H\"{o}lder-norm}} \leq C\norm{f}_{W^{1,p}(\Rl^n)}
        \end{align*}
        That is, in higher dimensions, it takes higher regularity to ensure continuity.
    \subsection{On $\Rl^n$, in general}
        Let $kp > n$.  Then
        \begin{align*}
            \norm{f}_{C^{k - \left\lfloor\frac{n}{p}\right\rfloor - 1}(\Rl^n)} \leq \underbrace{\norm{f}_{C^{k - \left\lfloor\frac{n}{p}\right\rfloor - 1, \gamma}(\Rl^n)}}_{\text{H\"{o}lder-norm}} \leq C\norm{f}_{W^{k,p}(\Rl^n)}
        \end{align*}

\section{Gagliardo-Nirenburg-Sobolev Inequalities}
    \subsection{On $\Rl^n$, with $k=1$}
        Let $1 \leq p < n$, and define $p^* \coloneqq \dfrac{np}{n - p}$.  Then $\forall f \in W^{1,p}(\Rl^n)$,
        \begin{align*}
            \norm{f}_{L^{p^*}(\Rl^n)} \leq C\norm{Df}_{L^p(\Rl^n)}
        \end{align*}
    \subsection{On $\Rl^n$, in general}
        Let $1 \leq kp < n$, and define $p^* \coloneqq \dfrac{np}{n - kp}$.  Then $\forall f \in W^{k,p}(\Rl^n)$,
        \begin{align*}
            \norm{f}_{L^{p^*}(\Rl^n)} \leq C\norm{D^kf}_{L^p(\Rl^n)}
        \end{align*}
        Furthermore, if $\Omega$ is open, bounded, and has $C^1$ boundary, then
        \begin{align*}
            \norm{f}_{L^{p^*}(\Omega)} \leq C\norm{f}_{W^{k,p}(\Omega)}
        \end{align*}

\section{Young's Inequality}
    Let $1 + \dfrac{1}{r} = \dfrac{1}{p} + \dfrac{1}{q}$.  Then
    \begin{align*}
        \norm{f * g}_r \leq \norm{f}_p\norm{g}_q
    \end{align*}

\section{H\"{o}lder's Inequality}
    Let $\dfrac{1}{p} + \dfrac{1}{q} = 1$.  Then
    \begin{align*}
        \norm{fg}_1 \leq \norm{f}_p\norm{g}_q
    \end{align*}

\section{Interpolation in Lebesgue Space}
    Let $\dfrac{1}{p} = \dfrac{a}{q} + \dfrac{1-a}{r}$.  Then
    \begin{align*}
        \norm{f}_p \leq \norm{f}_q^a + \norm{f}_r^{1-a}
    \end{align*}
    That is, functions in $L^q$ and $L^r$ are in $L^p$ for any $p \in (q,r)$.

\section{Tips and Tricks to Remember}
    \begin{itemize}
        \item Bounding a convolution?  Use Young's.
        \item Proving continuity or countinuous differentiability?  Use Sobolev Embeddings.
        \item Proving regularity?  Use Gagliardo-Nirenburg-Sobolev Inequalities.
        \item Can you prove a function in $L^1$ is bounded?  Use Interpolation.
        \item Bounding an integral?  Use H\"{o}lder's.
    \end{itemize}

\pagebreak

\section{Spectrum of Bounded Linear Operators on Hilbert Spaces}
    \subsection{In General}
        Let $A \in \mathcal{B}(\hilb)$.  Then
        \begin{itemize}
            \item $\sigma(A) \subset B_{\norm{A}}(0)$.
            \item $\sigma(A)$ is closed.
            \item $\sigma(A) \neq \emptyset$
            \item $r(A) = \lim_{n\rightarrow\infty}\norm{A^n}^{\frac{1}{n}}$.
            \item $\lambda \in \text{resi}(A) \implies \overline{\lambda}$ is an eigenvalue of $A^*$.
        \end{itemize}
    \subsection{For Self-Adjoint Operators}
        Let $A \in \mathcal{B}(\hilb)$ such that $A = A^*$.  Then
        \begin{itemize}
            \item $\sigma(A) \subset \Rl$.
            \item $\text{resi}(A) = \emptyset$.
            \item $r(A) = \norm{A}$.
            \item Eigenvectors correspoding to distinct eigenvalues are orthogonal.
        \end{itemize}
    \subsection{For Compact, Self-Adjoint Operators}
        Let $A \in \mathcal{B}(\hilb)$ such that $A = A^*$ and $A$ is compact.  Then
        \begin{itemize}
            \item $\sigma(A)$ consists entirely of eigenvalues, except possibly $0$, which may be in the continuous spectrum.
            \item Every nonzero eigenvalue has finite multiplicity, that is, the dimension of the eigenspace is finite, i.e.~$\dim\ker[A - \lambda I] < \infty$.
            \item If $\sigma(A)$ has an accumulation point, it must be $0$.  There are no other accumulation points.
            \item $A$ can be represented as a convergent (in operator norm) series of prejections onto eigenspaces.  That is,
            \begin{align*}
                A = \sum_{\lambda \in \sigma(A)}\lambda P_{\ker [A - \lambda I]}
            \end{align*}
        \end{itemize}
    
\end{document}
