\documentclass[avery5388,grid,frame]{flashcards}

\cardfrontstyle[\large\slshape]{headings}
\cardbackstyle{empty}

\usepackage{amssymb, amsmath, amsfonts}
\usepackage{mathtools}
\usepackage{physics}
\usepackage{enumerate}
\usepackage{array}

\newcommand{\E}{\varepsilon}
\newcommand{\ran}{\mathrm{ran}\,}
\newcommand{\supp}{\mathrm{supp}\,}
\newcommand{\dist}{\mathrm{dist}\,}
\newcommand{\esssupp}{\mathrm{ess}\,\mathrm{supp}\,}
\newcommand{\ind}{\mathrm{ind}\,}
\newcommand{\sgn}[1]{\mathrm{sgn}\left[#1\right]}
\newcommand{\f}[3]{#1\ :\ #2 \rightarrow #3}
\def\Rl{\mathbb{R}}
\def\Cx{\mathbb{C}}
\def\hilb{\mathcal{H}}
\def\torus{\mathbb{T}}

\begin{document}

\cardfrontfoot{Lieb and Loss Chapter 2}


\begin{flashcard}
    {Define $L^p$ space for $1 \leq p \leq \infty$.}
    $L^p$ is the space of all $p$\textsuperscript{th} power summable functions. \\

    Let $\Omega$ be a measure space with a positive measure $\mu$ and let $1 \leq p < \infty$.  Then
    \begin{align*}
        L^p(\Omega,\mu) \coloneqq \left\{f\ |\ \f{f}{\Omega}{\Cx},\ f \text{ is $\mu$-summable}\ \text{ and }\ \abs{f}^p \text{ is $\mu$-summable}\right\}.
    \end{align*}

    The norm of $L^p$ is given by $\displaystyle\norm{f}_{L^p} = \qty(\int_\Omega \abs{f}^p \dd\mu)^\frac{1}{p}$.

    For $p = \infty$,
    \begin{align*}
        L^\infty(\Omega,\mu) \coloneqq \{f\ |\ \f{f}{\Omega}{\Cx},\ f \text{is $\mu$-measurable and } \exists \text{ constant } K \\ \text{ such that } \abs{f(x)} < K \text{ for $\mu$ almost every } x \in \Omega \}
    \end{align*}
    with norm $\displaystyle\norm{f}_{L^\infty} = \inf\left\{K\ |\ \abs{f(x)} < K \text{ for $\mu$ almost every } x \in \Omega\right\}$.
\end{flashcard}

\begin{flashcard}
    {For functions in $L^p \cap L^\infty$, how is the $L^p$ norm related to the $L^\infty$ norm?}
    If $f \in L^p \cap L^\infty$, then $f \in L^q$ for all $q > p$ and $\norm{f}_\infty = \lim_{p \rightarrow \infty}\norm{f}_p$.
\end{flashcard}

\begin{flashcard}
    {What is a convex set?  Convex function?  What does it mean to be strictly convex?  Concave?  How are these related to continuity?}
    \begin{itemize}
        \item A convex set $K \subset \Rl^n$ is one for which $\lambda x + (1 - \lambda)y \in K$ for all $x,y\in K$ and $0\leq\lambda\leq1$.
        \item A convex function $f$ on a convex set $K$ is a real-valued function satisfying $f(\lambda x + (1 - \lambda)y) \leq \lambda f(x) + (1 - \lambda)f(y)$ for all $x,y \in K$ and $0\leq\lambda\leq1$.
        \item A function is strictly convex if equality never holds whenever $x \neq y$ and $0 < \lambda < 1$.
        \item A function is concave if the inequality is reversed.
        \item If $K$ is open then convex functions are continuous.
    \end{itemize}
\end{flashcard}

\begin{flashcard}
    {What is Jensen's Inequality?}
    Let $\f{J}{\Rl}{\Rl}$ be a convex function and $f$ a real-valued function on some finite measurable set $\Omega$.  Define $\langle\cdot\rangle$ to be the average of a function, i.e.
    \begin{align*}
        \langle f \rangle \coloneqq \frac{1}{\mu(\Omega)}\int_\Omega f.
    \end{align*}
    Then
    \begin{enumerate}[ (i)]
        \item $\qty[J \circ f]_- \in L^1(\Omega)$;
        \item $\langle F \circ f \rangle \geq J(\langle f \rangle)$.
    \end{enumerate}
    In English,
    \begin{enumerate}[(i)]
        \item The negative part of the composition is absolutely summable;
        \item The average of the composition is at least the composition of the average.
    \end{enumerate}
\end{flashcard}

\begin{flashcard}
    {What is H\"{o}lder's Inequality?  What is the Schwarz Inequality?  What is the generalization for $m$ functions?}
    Let $1\leq p\leq\infty$ and let $q$ be the dual index of $p$.  Then if $f \in L^p$ and $g \in L^q$, then $fg \in L^1$ and
    \begin{align*}
        \norm{fg}_1 \leq \norm{f}_p\norm{g}_q.
    \end{align*}

    The Schwarz Inequality is the special case when $p = q = 2$.  We have
    \begin{align*}
        \norm{fg}_1 \leq \norm{f}_2\norm{g}_2
    \end{align*}

    To generalize, for $i = 1, 2, \dots, n$, let $f_i \in L^{p_i}$ and $\displaystyle\frac{1}{p_1} + \dots + \frac{1}{p_n} = 1$.  Then
    \begin{align*}
        \prod_{i=1}^n f_i \in L^1 \qquad \text{and} \qquad \norm{\prod_{i=1}^n f_i}_1 \leq \prod_{i=1}^n\norm{f_i}_{p_i}
    \end{align*}
\end{flashcard}

\begin{flashcard}
    {What is Hanner's Inequality?}
    Let $f,g \in L^p$.  If $1 \leq p \leq 2$, then (Parallelogram Identity)
    \begin{align*}
        \norm{f + g}_p^p + \norm{f - g}_p^p \leq \qty(\norm{f}_p + \norm{g}_p)^p + \abs{\norm{f}_p - \norm{g}_p}^p
    \end{align*}
    and
    \begin{align*}
        \qty(\norm{f + g}_p + \norm{f - g}_p)^p + \abs{\norm{f + g}_p - \norm{f - g}_p}^p \leq 2^p\qty(\norm{f}_p^p + \norm{g}_p^p).
    \end{align*}
    If $2 \leq p < \infty$, the inequalities are reversed.  
\end{flashcard}

\begin{flashcard}
    {Is $L^p$ complete (and thus Banach)?}
    Yes.  Let $1 \leq p \leq \infty$ and let $(f_i)$ be a Cauchy sequence in $L^p$, i.e.~$\norm{f_i - f_j}_p \rightarrow 0$ as $i,j \rightarrow \infty$.  Then there is a unique function $f \in L^p$ such that $\norm{f_i - f}_p \rightarrow 0$ as $i \rightarrow \infty$, i.e.
    \begin{align*}
        f_i \rightarrow f \qquad \text{say ``$f_i$ converges strongly to $f$''.}
    \end{align*}
\end{flashcard}

\begin{flashcard}
    {State the projection theorem for convex subsets of $L^p$.}
    Let $1 < p < \infty$ and let $K$ be a convex subset of $L^p$.  Let $f \in L^p$ such that $f \not\in K$ and define
    \begin{align*}
        D \coloneqq \dist(f,K) \inf_{g \in K}\norm{f - g}_p.
    \end{align*}
    Then $\exists h \in K$ such that
    \begin{align*}
        \norm{f - h}_p = D.
    \end{align*}
\end{flashcard}

\begin{flashcard}
    {Define weak convergence in $L^p$.}
    Let $(f_i)$ be a sequence in $L^p$.  If $L(f_i) \rightarrow L(f)$ for every bounded linear functional $L$ on $L^p$, then we say $f_i \rightharpoonup f$, or $f_i$ weakly converges to $f$.

    It can be shown that for $1 \leq p < \infty$, $(L^p)^* \cong L^q$, where $q$ is the dual index of $p$, and that every bounded linear functional $L \in (L^p)^*$ can be represented as integration against a unique $L^q$ function, i.e.~$\forall L \in (L^p)^*$, $\exists!g\in L^q$ such that
    \begin{align*}
        L(f) = \int f g
    \end{align*}
    for every $f \in L^p$.  Thus, $(f_i)$ converges weakly in $L^p$ if
    \begin{align*}
        \int f_i g \rightarrow \int f g
    \end{align*}
    for every $g \in L^q$.
\end{flashcard}

\begin{flashcard}
    {State the ``Linear functionals separate'' theorem.}
    Suppose $f \in L^p$ with $L(f) = 0$ for all $L \in (L^p)^*$.  Then $f = 0$. \\

    Consequently, if $f_i \rightharpoonup g$ and $f_i \rightharpoonup h$, then $g = h$.
\end{flashcard}

\begin{flashcard}
    {State the Uniform Boundedness Principle.}
    Let $(f_i)$ be a sequence in $L^p$ such that $\forall L \in (L^p)^*$ the sequence $(L(f_i))$ is bounded in $\Cx$.  Then $\qty(\norm{f_i}_p)$ is a bounded sequence in $\Rl$.
\end{flashcard}

\begin{flashcard}
    {Define convolution for functions on $\Rl^n$.}
    For $\f{f,g}{\Rl^n}{\Cx}$, we define the convolution of $f$ and $g$, denoted $f * g$, as
    \begin{align*}
        (f*g)(x) \coloneqq \int_{\Rl^n} f(x - y)g(y)\dd y.
    \end{align*}
\end{flashcard}

\begin{flashcard}
    {Define a mollification.  Why are they useful?}
    Let $j \in L^1(\Rl^n)$ with $\int_{\Rl^n}j = 1$.  For $\E > 0$, define $j_\E$ as
    \begin{align*}
        j_\E(x) \coloneqq \frac{1}{\E^n}j\qty(\frac{x}{\E}).
    \end{align*}
    so that $\norm{j_E}_1 = \norm{j}_1$ and $\int_{\Rl^n}j_\E = 1$.

    Define the mollification of a function $f \in L^p(\Rl^n)$ for some $1 \leq p < \infty$, denoted $f_\E$, as the convolution of $f$ and $j_\E$ for some $\E$, that is,
    \begin{align*}
        f_\E = f * j_\E.
    \end{align*}

    Then $f_\E \in L^p(\Rl^n)$ and $\norm{f_\E}_p \leq \norm{f}_p\norm{j}_1$.  Also, $f_\E \rightarrow f$ strongly in $L^p$, that is, $\norm{f_\E - f}_p \rightarrow 0$. \\

    In addition, if $j \in C_C^\infty$, then $f_\E \in C^\infty$.  This is a concrete construction which shows that $C^\infty$ functions are dense in $L^p$.
\end{flashcard}

\begin{flashcard}
    {Is $L^p(\Rl^n)$ separable?  What does that mean?}
    Yes, $L^p(\Rl^n)$ is separable.  This means there is a countable dense subset of $L^p(\Rl^n)$, that is, $\exists \Phi = \{\phi_1,\phi_2,\dots\} \subset L^p(\Rl^n)$ such that $\forall f \in L^p$ and $\E > 0$, $\exists \phi_j \in \Phi$ such that $\norm{f - \phi_j}_p < \E$.
\end{flashcard}

\begin{flashcard}
    {State the Banach-Alaoglu Theorem.}
    Let $\Omega \subset \Rl^n$ be a measurable set and consider $L^p(\Omega)$ with $1 < p < \infty$.  Let $(f_i)$ be a bounded sequence in $L^p$.  Then there exists a subsequence $(f_{i_j})$ and $f \in L^p$ such that $f_{i_j} \rightharpoonup f$ in $L^p$.  That is, bounded sets in $L^p$ are weakly compact.
\end{flashcard}

\begin{flashcard}
    {How does Urysohn's Lemma give that $C_C^\infty$ is dense in $L^p$?}
    Let $\Omega \subset \Rl^n$ be an open set and let $K \subset \Omega$ be compact.  Then there is a functin $J_K \in C_C^\infty(\Omega)$ such that $0 \leq J_K(x) \leq 1$ for all $x \in \Omega$ and $J_K(x) = 1$ for all $x \in K$. \\

    As a consequence, there is a sequence of functions $(g_i) \in C_C^\infty$ that take values in $[0,1]$ and such that $\displaystyle\lim_{j\rightarrow\infty}g_j(x) = 1$ for every $x \in \Omega$. \\

    As a second consequence, given a sequence of functions $(f_i) \in C^\infty$ such that $f_i$ converges strongly to a function $f \in L^p$, the sequence $(h_i) = (g_if_i) \in C_C^\infty$ and $h_i \rightarrow f$ strongly. \\

    This shows, since $C^\infty$ is dense in $L^p$, that $C_C^\infty$ is dense in $L^p$.
\end{flashcard}

\begin{flashcard}
    {What is special about convolutions of functions in dual $L^p$ spaces?}
    If $f \in L^p(\Rl^n)$ and $g \in L^q(\Rl^n)$, where $q$ is the dual index of $p$, then $f*g$ is continuous and $(f*g)$ tends to $0$ at infinity.
\end{flashcard}

\end{document}

