\documentclass[avery5388,grid,frame]{flashcards}

\cardfrontstyle[\large\slshape]{headings}
\cardbackstyle{empty}

\usepackage{amssymb, amsmath, amsfonts}
\usepackage{mathtools}
\usepackage{physics}
\usepackage{enumerate}

\newcommand{\E}{\varepsilon}
\def\Rl{\mathbb{R}}
\def\Cx{\mathbb{C}}
\def\hilb{\mathcal{H}}
\def\torus{\mathbb{T}}

\begin{document}

\cardfrontfoot{Applied Analysis Chapter 7}


\begin{flashcard}
    {What is the Fourier basis for $L^2(\torus)$?}
    The Fourier basis is $\left\{e_n\ | n \in \mathbb{Z}\right\}$ where
    \begin{align*}
        e_n(x) = \frac{1}{\sqrt{2\pi}}e^{inx}.
    \end{align*}
\end{flashcard}

\begin{flashcard}
    {What are trigonometric polynomials?  Why are they important?}
    A ``trigonometric polynomial'' is a special name for finite linear combinations of the Fourier basis.  They are important since they are dense in $C(\torus)$.  Proving they are dense in $C(\torus)$, along with showing orthonormality, proves the Fourier basis is indeed a basis of $L^2(\torus)$.
\end{flashcard}

\begin{flashcard}
    {Define a convolution of two continuous functions}
    The convolution of two continuous functions $f,g\ :\ \torus \rightarrow \Cx$, denoted $f * g$, is a continuous function defined by the following integral:
    \begin{align*}
        (f*g)(x) = \int_\torus f(x-y)g(y)\dd y = \int_\torus f(y)g(x-y)\dd y
    \end{align*}
\end{flashcard}

\begin{flashcard}
    {What is an approximate identity?  Why are they important?}
    A family of functions $\{\phi_n \in C(\torus)\ |\ n \in \mathbb{N}\}$ is an approximate identity if
    \begin{enumerate}[(a)]
        \setlength{\itemsep}{0pt}
        \item $\displaystyle\phi_n(x) \geq 0$;
        \item $\displaystyle\int_\torus \phi_n(x)\dd x = 1$ for every $n \in \mathbb{N}$;
        \item $\displaystyle\lim_n\rightarrow \infty \int_{\delta\leq\abs{x}\leq \pi} \phi_n(x) \dd x = 0$ for every $0 < \delta \leq \pi$.
    \end{enumerate}
    In English:
    \begin{enumerate}[(a)]
        \setlength{\itemsep}{0pt}
        \item All functions are non-negative.
        \item The area under each curve is $1$.
        \item As $n \rightarrow \infty$, most of the area accumulates near $0$.
    \end{enumerate}
    Approximate identities are important since for large $n$, the convolution of $f$ with $\phi_n$ gives a local average of $f$.
\end{flashcard}

\begin{flashcard}
    {If $\{\phi_n \in C(\torus)\ |\ n \in \mathbb{N}\}$ is an approximate identity and $f \in C(\torus)$, how does $\phi_n * f$ converge to $f$?  Uniformly, or just pointwise?}
    Uniformly.  The proof uses:
    \begin{itemize}
        \item $f(x) = \int_\torus \phi_n(y)f(x)\dd y$ (clever since $f(x)$ is independent of $y$)
        \item Splitting the integral in to $y$ in a $\delta$-ball around $0$ and $y$ outside of that $\delta$-ball
        \item Different components of the product are small on different subsets of $\torus$.
    \end{itemize}
\end{flashcard}

\begin{flashcard}
    {The trigonometric polynomials are dense in $C(\torus)$ with respect to the uniform norm.}
    The proof uses:
    \begin{itemize}
        \item A specific approximate identity: $\phi_n(x) = c_n(1 + \cos x)^n$, where $c_n$ is chosen so $\int_\torus \phi_n = 1$.
        \item Since $\phi_n * f \rightarrow f$ uniformly, just show it is in fact a trigonometric polynomial.
        \item Noting $\phi_n$ is a trigonometric polynomial: $\phi_n(x) = \sum_{k=-n}^n a_{n,k}e^{ikx}$ where $2^{-n}c_n\binom{2n}{n+k}$.
    \end{itemize}
\end{flashcard}

\begin{flashcard}
    {Why can any $f \in L^2(\torus)$ be expanded as a Fourier series?  What are its coefficients?}
    Since trigonometric polynomials are dense in $C(\torus)$, which is dense in $L^2(\torus)$, any $f \in L^2(\torus)$ can be written as
    \begin{align*}
        f(x) = \sum_{n=-\infty}^\infty \hat{f}_ne_n(x)
    \end{align*}
    where $e_n(x) = \frac{1}{\sqrt{2\pi}}e^{inx}$, and so
    \begin{align*}
        f(x) = \frac{1}{2\pi}\sum_{n=-\infty}^\infty \hat{f}_ne^{inx}.
    \end{align*}
    The orthonormality of $\{e_n\}$ provides a concrete calculation of $\hat{f}_n$:
    \begin{align*}
        \hat{f}_n = (e_n, f) = \frac{1}{\sqrt{2\pi}}\int_\torus f(x) e^{-inx}\dd x.
    \end{align*}
\end{flashcard}

\begin{flashcard}
    {What is Parseval's Identity and why is it important in Fourier theory?}
    Parseval's Identity is $(f,g) = \sum_{n\in\mathbb{N}} \overline{f_n}g_n$ where $f_n$ and $g_n$ are the coefficients of the expansions of $f$ and $g$ with respect to an orthonormal basis.  In Fourier theory, we have
    \vspace{-5pt}
    \begin{align*}
        \int_\torus \overline{f(x)}g(x)\dd x = (f,g) = \sum_{n=-\infty}^\infty \overline{\hat{f}_n}\hat{g}_n
    \end{align*}
    \vspace{-5pt}
    Taking $g = f$ gives
    \vspace{-5pt}
    \begin{align*}
        \norm{f}_{L^2(\torus)}^2 = \int_\torus \abs{f(x)}^2\dd x = \sum_{n=-\infty}^\infty \abs{\hat{f}_n}^2 = \norm{\qty(\hat{f}_n)}_{\ell^2(\mathbb{Z})}^2
    \end{align*}
    and thus we can define a Hilbert space isomorphism $\mathcal{F}$ between $L^2(\torus)$ and $\ell^2(\torus)$ by
    \begin{align*}
        \mathcal{F}f = \qty(\hat{f}_n)_{n=-\infty}^\infty.
    \end{align*}
\end{flashcard}

\begin{flashcard}
    {If $f,g \in L^2(\torus)$, then $f*g$ is continuous ($f*g \in C(\torus)$) and
    \begin{align*}
        \norm{f * g}_\infty \leq \norm{f}_2\norm{g}_2
    \end{align*}}
    The proof uses:
    \begin{itemize}
        \item The Cauchy-Schwarz inequality applied to the definition of convolution
        \item The convolution of continuous functions in continuous
        \item Forming continuous approximations of $f$ and $g$ (sequences of continuous functions approaching $f$ and $g$)
        \item Triangle Inequality
        \item Completeness of $C(\torus)$
    \end{itemize}
    This is a special case of Young's inequality.
\end{flashcard}

\begin{flashcard}
    {State and prove the Convolution Theorem}
    For $f,g \in L^2(\torus)$, $\widehat{(f*g)}_n = \sqrt{2\pi}\hat{f}_n\hat{g}_n$.

    First let $f,g \in C(\torus)$.  Then
    \vspace{-5pt}
    \begin{align*}
        \widehat{(f*g)}_n = \frac{1}{\sqrt{2\pi}}\int_\torus (f*g)(x) e^{-inx}\dd x = \frac{1}{\sqrt{2\pi}}\int_\torus\qty(\int_\torus f(x - y)g(y)\dd y) e^{-inx} \dd x
    \end{align*}
    \vspace{-5pt}
    We can use Fubini's Theorem since the integrand is continuous:
    \vspace{-5pt}
    \begin{align*}
        \widehat{(f*g)}_n = \int_\torus\frac{1}{\sqrt{2\pi}}\qty(\int_\torus f(x - y)e^{-in(x - y)} \dd x)g(y)e^{iny} \dd y
    \end{align*}
    \vspace{-5pt}
    by multiplying and dividing by $e^{iny}$.  Then
    \vspace{-5pt}
    \begin{align*}
        \widehat{(f*g)}_n = \hat{f}_n\int_\torus g(y)e^{-iny}\dd y = \sqrt{2\pi}\hat{f}_n\hat{g}_n
    \end{align*}
    By density of $C(\torus)$ in $L^2(\torus)$, and continuity of the Fourier transform and the convolution with respect to $L^2$ convergence, this result holds for $L^2(\torus)$ functions too.
\end{flashcard}

\begin{flashcard}
    {What are other useful bases of $L^2(\torus)$?}
    $\left\{e_n(x) = e^{inx}\ |\ n \in \mathbb{Z}\right\}$ is an orthogonal (not orthonormal) basis.  Then
    \begin{align*}
        f(x) = \sum_{n=-\infty}^\infty \hat{f}_n e^{inx}, \qquad \text{where}\qquad \hat{f}_n = \frac{1}{2\pi}\int_\torus f(x) e^{-inx}\dd x
    \end{align*}
    $\{1, \cos(nx), \sin(nx)\ |\ n = 1, 2, \dots\}$ is an orthogonal basis.  Then
    \begin{align*}
        f(x) = \frac{1}{2}a_0 + \sum_{n=1}^\infty \qty[a_n\cos(nx) + b_n \sin(nx)]
    \end{align*}
    where $a_n = \frac{1}{\pi}\int_\torus f(x) \cos(nx)\dd x$ and $b_n(x) = \frac{1}{\pi}\int_\torus \sin(nx) \dd x$.  These are useful since odd(even) functions have $\sin$($\cos$) expansions.  Finally, any function defined on $[0,\pi]$ can be extended to an odd(even) function on $[-\pi,\pi]$ and can thus be represented by a $\sin$($\cos$) expansion.
\end{flashcard}

\begin{flashcard}
    {Generalize the Fourier series to a $d$-dimensional $2\pi$-periodic function.}
    A function $f\ :\ \Rl^d \rightarrow \Cx$ is $2\pi$-periodic in each variable if
    \begin{align*}
        f(x_1, x_2, \dots, x_i + 2\pi, \dots, x_d) = f(x_1, x_2, \dots, x_i, \dots, x_d) \qquad \text{for } i = 1, \dots, d.
    \end{align*}
    An orthonormal basis of $L^2(\torus^d)$ is
    \begin{align*}
        e_\textbf{n}(\textbf{x}) = \frac{1}{(2\pi)^{d/2}}e^{i \textbf{n}\cdot \textbf{x}}, \qquad \text{where } \textbf{x} \in \torus^d,\ \textbf{n} \in \mathbb{Z}^d,\ \text{and } \textbf{n}\cdot\textbf{x} = \sum_{i=1}^d n_ix_i.
    \end{align*}
    The Fourier series expansion of a function $f \in L^2(\torus^d)$ is
    \begin{align*}
        f(\textbf{x}) = \frac{1}{\qty(2\pi)^{d/2}}\sum_{\textbf{n}\in \mathbb{Z}^d}\hat{f}_\textbf{n}e^{i \textbf{n}\cdot \textbf{x}}, \qquad \text{where } \hat{f}_\textbf{n} = \frac{1}{\qty(2\pi)^{d/2}}\int_{\torus^d}f(\textbf{x})e^{-i \textbf{n}\cdot \textbf{x}}\dd \textbf{x}
    \end{align*}
\end{flashcard}

\begin{flashcard}
    {What is the connection between smoothness of a function and its Fourier coefficients?}
    The smoother the function, the faster its Fourier coefficients decay.  Smooth functions contain small amounts of high-frequency components.
\end{flashcard}

\begin{flashcard}
    {How is differentiation made easier by fourier series?  How does this expand the notion of derivative?}
    The Fourier coefficients of a derivative is a scalar multiple (in particular, $in$) of the Fourier coefficients of the original function.
    \begin{align*}
        \widehat{f'}_n = \frac{1}{\sqrt{2\pi}}\int_0^{2\pi} e^{inx}f'(x)\dd x.
    \end{align*}
    Integration by parts gives
    \begin{align*}
        \widehat{f'}_n = \frac{1}{\sqrt{2\pi}}\qty[f(2\pi) - f(0)] - \frac{1}{\sqrt{2\pi}}\int_0^{2\pi}(-in)e^{-inx}f(x) \dd x = in\hat{f}_n.
    \end{align*}
    Induction gives $\widehat{f^{(k)}}_n = (in)^k\hat{f}_n$.  Heuristically, derivatives are a ``roughing'' operation, adding higher amounts of high-frequency components.

    We can take a ``derivative'' of an arbitrary $L^2$ function by transforming it into an $\ell^2$ sequence and multiplying each component by $in$.  This leads to what is called a ``weak derivative.''
\end{flashcard}

\begin{flashcard}
    {Define Sobolev space on $\torus$ in terms of Fourier coefficients.}
    The Sobolev space $H^1(\torus) = W^{1,2}(\torus)$ consists of all functions
    \begin{align*}
        f(x) = \frac{1}{\sqrt{2\pi}}\sum_{n \in \mathbb{N}}\hat{f}_n e^{inx} \in L^2(\torus) \qquad \text{such that} \qquad \sum_{n \in \mathbb{N}} n^2 \abs{\hat{f}_n}^2 < \infty.
    \end{align*}
    That is, all functions $f \in L^2(\torus)$ such that the weak derivative of $f$, denoted $f'$, is also in $L^2(\torus)$.  $f'$ is given by $\displaystyle f'(x) = \frac{1}{\sqrt{2\pi}}\sum_{n\in\mathbb{N}}in\hat{f}_n e^{inx}$.
    The inner product on $H^1(\torus)$ is given by
    \begin{align*}
        (f, g)_{H^1(\torus)} = \int_\torus \qty[\overline{f(x)}g(x) + \overline{f'(x)}g'(x)]\dd x = (f, g)_{L^2(\torus)} + (f', g')_{L^2(\torus)},
    \end{align*}
    and Parseval's Theorem gives $\displaystyle (f,g)_{H1(\torus)} = \sum_{n\in\mathbb{N}}(1 + n^2)\overline{\hat{f}_n}\hat{g}_n$.
\end{flashcard}

\begin{flashcard}
    {Use Parseval's Theorem to show that weak derivatives satisfy integration by parts}
    \begin{align*}
        \int_\torus f'g \dd x = (\overline{f'},g)_{L^2} = \sum_{n\in\mathbb{N}}-in\overline{\hat{f}_n}\hat{g}_n = -\sum_{n\in\mathbb{N}}\overline{\hat{f}_n}in\hat{g}_n = -(f,g')_{L^2} = -\int_\torus fg' \dd x
    \end{align*}
\end{flashcard}

\begin{flashcard}
    {Define a weak derivative of an $L^2$ function using test functions and the theory of bounded linear operators on Hilbert spaces.}
    Let $f \in H^1(\torus)$ and define the bounded linear functional $F\ :\ C^1(\torus) \subset L^2(\torus) \rightarrow \Cx$ by
    \begin{align*}
        F(\phi) = -\int_\torus f \phi' \dd x.
    \end{align*}
    Since $F$ is a bounded linear functional defined on $C^1(\torus)$ and $C^1(\torus)$ is dense in $H^1(\torus)$, then we can uniquely extend $F$ to a bounded linear functional on $H^1(\torus)$.
\end{flashcard}

\end{document}

