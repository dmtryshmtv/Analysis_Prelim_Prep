\documentclass[avery5388,grid,frame]{flashcards}

\cardfrontstyle[\large\slshape]{headings}
\cardbackstyle{empty}

\usepackage{amssymb, amsmath, amsfonts}
\usepackage{mathtools}
\usepackage{physics}
\usepackage{enumerate}
\usepackage{array}

\newcommand{\E}{\varepsilon}
\newcommand{\ran}{\mathrm{ran}\,}
\newcommand{\ind}{\mathrm{ind}\,}
\newcommand{\Lip}{\mathrm{Lip}\,}
\newcommand{\supp}{\mathrm{supp}\,}
\newcommand{\sgn}[1]{\mathrm{sgn}\left[#1\right]}
\newcommand{\f}[3]{#1\ :\ #2 \rightarrow #3}
\def\Rl{\mathbb{R}}
\def\Cx{\mathbb{C}}
\def\hilb{\mathcal{H}}
\def\torus{\mathbb{T}}

\begin{document}

\cardfrontfoot{Applied Analysis Chapter 3}


\begin{flashcard}
    {What is a contraction mapping?}
    Let $\f{T}{(X,d)}{(X,d)}$ be a map on a metrix space $X$.  We say $T$ is a contraction mapping (or just contraction) if there is a small constant $c \in [0,1)$ such that
    \begin{align*}
        d(T(x),T(y)) \leq c d(x,y) \qquad \forall x,y \in X.
    \end{align*}
    Contractions maps points closer together. \\

    Contractions are uniformly continuous.
\end{flashcard}

\begin{flashcard}
    {What is a fixed point of a map $T$?  What is the relationship between contraction mappings and fixed points?}
    A point $x \in X$ is called a fixed point of $T$ if $Tx = x$. \\

    Every contraction mapping has a unique fixed point.
\end{flashcard}

\begin{flashcard}
    {How is the Contraction Mapping Theorem applicable to Fredholm Integral Operators?}
    Define a map $\f{T}{C([a,b]}{C([a,b]})$ by $Tf = g + Kf$ where
    \begin{align*}
        (Kf)(x) = \int_a^b k(x,y)f(y)\dd y
    \end{align*}
    for some $k \in C([a,b]^2)$ with ``max row sum'' less than $1$, i.e. $\displaystyle\max_{a\leq x\leq b}\left\{\int_a^b\abs{k(x,y)}\dd y\right\} < 1$.  Then for any $g \in C([a,b])$, there is a unique $f \in C([a,b])$ such that $(I - Kj)f = g$. \\

    This is shown by proving $T$ is a contraction mapping, which shows that there is a unique $f \in C([a,b])$ such that $Tf = f$.
\end{flashcard}

\end{document}

