\documentclass[avery5388,grid,frame]{flashcards}

\cardfrontstyle[\large\slshape]{headings}
\cardbackstyle{empty}

\usepackage{amssymb, amsmath, amsfonts}
\usepackage{mathtools}
\usepackage{physics}
\usepackage{enumerate}
\usepackage{array}

\newcommand{\E}{\varepsilon}
\newcommand{\ran}{\mathrm{ran}\,}
\newcommand{\ind}{\mathrm{ind}\,}
\newcommand{\Lip}{\mathrm{Lip}\,}
\newcommand{\supp}{\mathrm{supp}\,}
\newcommand{\sgn}[1]{\mathrm{sgn}\left[#1\right]}
\newcommand{\f}[3]{#1\ :\ #2 \rightarrow #3}
\def\Rl{\mathbb{R}}
\def\Cx{\mathbb{C}}
\def\hilb{\mathcal{H}}
\def\torus{\mathbb{T}}
\def\topo{\mathcal{T}}
\def\sopo{\mathcal{S}}

\begin{document}

\cardfrontfoot{Shkoller Analysis Chapter 2.1}


\begin{flashcard}
    {State the Divergence Theorem}
    Let $\Omega \subset \Rl^n$ be a Lipschitz domain and $w = (w_1, \dots, w_n) \in \mathcal{C}^1(\overline{\Omega})$ with outward pointing normal $N$.  Then
    \begin{align*}
        \int_\Omega \div w\ \dd x = \int_{\partial \Omega} w \cdot N \dd S
    \end{align*}
\end{flashcard}

\begin{flashcard}
    {State Green's First and Second Identities.}
    Let $\Omega$ be a smooth domain and let $u \in \mathcal{C}^2(\overline{\Omega})$ and $v \in \mathcal{C}^1(\overline{\Omega})$-funtions.  Then we have Green's First Identity:
    \begin{align*}
        \int_\Omega \grad v \cdot \grad u + v\laplacian u \dd x = \int_\Omega \div\qty(v\grad u)\dd x = \int_{\partial \Omega} v \frac{\partial u}{\partial N}\dd S
    \end{align*}
    Exchanging $u$ and $v$ in Green's First Identity and finding the difference gives Green's Second Identity:
    \begin{align*}
        \int_\Omega \qty(v \laplacian u - u\laplacian v)\dd x = \int_{\partial \Omega}\qty[v\frac{\partial u}{\partial N} - u \frac{\partial v}{\partial N}]\dd S
    \end{align*}
\end{flashcard}

\begin{flashcard}
    {What is a ``test function''?}
    Test functions are smooth functions with compact support, i.e.
    \begin{align*}
        \mathcal{C}_c^\infty = \left\{u \in \mathcal{C}^\infty\ :\ \supp u \subset \mathcal{V} \Subset \Omega\right\}.
    \end{align*}
\end{flashcard}

\begin{flashcard}
    {What is a weak derivative of an $L^1_{\text{loc}}$ function?}
    Let $u \in L^1_{\text{loc}}(\Omega)$.  Then $v^\alpha \in L^1_\text{loc}(\Omega)$ is called the $\alpha^\text{th}$ weak derivative of $u$, written $v^\alpha = D^\alpha u$, if
    \begin{align*}
        \int_\Omega u(x) D^\alpha \phi(x) \dd x = (-1)^{\abs{\alpha}}\int_\Omega  v^\alpha(x)\phi(x)\dd x \qquad \forall \phi \in \mathcal{C}_c^\infty(\Omega),
    \end{align*}
    where $\alpha \in \mathbb{N}^n$ is a multi-index with $\abs{\alpha} = \alpha_1 + \dots + \alpha_n$.
\end{flashcard}

\begin{flashcard}
    {Why does $f$ have a weak-derivative, but $g$ does not?
    \begin{equation*}
        \begin{aligned}
            f(x) = \begin{cases}
                x & \text{ if } x \in (0,1) \\
                1 & \text{ if } x \in (1,2)
            \end{cases}
        \end{aligned}
    \end{equation*}
    \begin{equation*}
        \begin{aligned}
            g(x) = \begin{cases}
                x & \text{ if } x \in (0,1) \\
                2 & \text{ if } x \in (1,2)
            \end{cases}
        \end{aligned}
    \end{equation*}}
    We can explicitly calculate the weak derivative (using integration by parts) of $f$.  For $g$, however, assuming a weak derivative exists results in a contradiction by exploiting the boundary terms in the integration by parts that don't cancel each other out.
\end{flashcard}

\begin{flashcard}
    {Define $W^{1,p}(\Omega)$ for $1 \leq p \leq \infty$.  Then define $W^{k,p}(\Omega)$ for $1 \leq p \leq \infty$ and $k \in \mathbb{N}$.  What is the norm in $W^{k,p}$?}
    \begin{align*}
        W^{1,p}(\Omega) = \left\{u \in L^p(\Omega)\ |\ \text{the weak derivative $u'$ of $u$ exists, and } u' \in L^p(\Omega)\right\}
    \end{align*}
    \begin{align*}
        W^{k,p}(\Omega) = \left\{u \in L^1_\text{loc}(\Omega)\ |\ D^\alpha u \text{ exists and is in } L^p(\Omega) \text{ for } \abs{\alpha} \leq k\right\}
    \end{align*}
    \begin{align*}
        \norm{u}_{W^{k,p}(\Omega)} = \qty(\sum_{\abs{\alpha}\leq k}\norm{D^\alpha u}_{L^p(\Omega)}^p)^{\frac{1}{p}}
    \end{align*}
    \begin{align*}
        \norm{u}_{W^{k,\infty}(\Omega)} = \sum_{\alpha \leq k} \norm{D^\alpha u}_{L^\infty(\Omega)}
    \end{align*}
\end{flashcard}

\begin{flashcard}
    {What is the Sobolev Embedding Theorem for $n=2$?}
    Only consider $\Rl^2$. \vspace{0.4cm}

    Let $kp > 2$ and $u \in \mathcal{C}_c^\infty(\Rl^2)$.  Then
    \begin{align*}
        \norm{u}_{L^\infty(\Rl^2)} \leq C \norm{u}_{W^{k,p}(\Rl^2)}
    \end{align*}
    In English, $W^{k,p}$ functions are bounded in $\Rl^2$.
\end{flashcard}

\begin{flashcard}
    {How can one explicitly approximate $W^{k,p}$ functions by smooth functions?}
    Choose $u \in W^{k,p}(\Omega)$, and let $(\eta_\E{n})$ be the standard mollifiers (increasingly concentrated Gaussian curves).  For each $\E$, define $$u^\E = \eta_\E * u$$ and $$\Omega_\E = \left\{x \in \Omega\ |\ \text{dist}(x,\partial\Omega)> \E\right\}.$$  Then
    \begin{enumerate}[(A)]
        \item $u^\E \in C^\infty(\Omega_\E)$ for each $\E > 0$, and
        \item $u^\E \rightarrow u$ in $W^{k,p}_\text{loc}(\Omega)$ as $\E \rightarrow 0$.
    \end{enumerate}
\end{flashcard}

\begin{flashcard}
    {Is it true that $\mathcal{C}^\infty(\Omega)$ is dense in $W^{k,p}(\Omega)$? \\ \vspace{.2cm}
    Is it true that $\mathcal{C}^\infty(\overline{\Omega})$ is dense in $W^{k,p}(\overline{\Omega})$?}
    Yes and yes.
\end{flashcard}

\begin{flashcard}
    {What is the norm in $\mathcal{C}^0$?  What is the norm in $\mathcal{C}^1$?  What is the is the space $\mathcal{C}^{0,\gamma}$?  What is its norm?  Is it Banach?}
    $\mathcal{C}^0$ is the space of continuous functions and $\mathcal{C}^1$ is the space of continuously differentiable functions.  They both are Banach spaces and their norms are
    \begin{equation*}
        \begin{aligned}
            \norm{u}_{\mathcal{C}^0(\overline{\Omega})} &= \sup_{x\in\overline{\Omega}}\norm{u(x)} \\
            \norm{u}_{\mathcal{C}^1(\overline{\Omega})} &= \norm{u}_{\mathcal{C}^0(\overline{\Omega})} + \norm{Du}_{\mathcal{C}^0(\overline{\Omega})}
        \end{aligned}
    \end{equation*}
    The space $\mathcal{C}^{0,\gamma}(\overline{\Omega})$ is called a H\"{o}lder space.  The H\"{o}lder norm is
    \vspace{-.2cm}
    \begin{align*}
        \norm{u}_{\mathcal{C}^{0,\gamma}(\overline{\Omega})} = \norm{u}_{\mathcal{C}^0(\overline{\Omega})} + \big[u\big]_{\mathcal{C}^{0,\gamma}(\overline{\overline{\Omega}})}
    \end{align*}
    where
    \vspace{-.2cm}
    \begin{align*}
        \big[u\big]_{\mathcal{C}^{0,\gamma}(\overline{\Omega})} = \max_{\substack{x,y\in\Omega\\x\neq y}}\qty(\frac{\abs{u(x) - u(y)}}{\abs{x - y}^\gamma}).
    \end{align*}
    \vspace{-.2cm}
    H\"{o}lder spaces are Banach.
\end{flashcard}

\begin{flashcard}
    {What is Morrey's Inequality?}
    Let $B_r\subset\Rl^n$ be a ball of radius $r$, and let $n < p \leq \infty$.  For $x,y\in B_r$, we have Morrey's Inequality:
    \begin{align*}
        \abs{u(x) - u(y)} \leq C\abs{x - y}^{1 - \frac{n}{p}}\norm{Du}_{L^p(B_r)} \qquad \forall u \in W^{1,p}(B_r).
    \end{align*}
\end{flashcard}

\begin{flashcard}
    {What is the Sobolev Embedding Theorem for $k=1$?  How is it an extension of the Sobolev Embedding Theorem for $n=2$?}
    The Sobolev Embedding Theorem is: there is a constant $C = C(p,n)$ such that
    \begin{align*}
        \norm{u}_{\mathcal{C}^{0,1-\frac{n}{p}}(\Rl^n)} \leq C\norm{u}_{W^{1,p}(\Rl^n)} \qquad \forall u \in W^{1,p}(\Rl^n).
    \end{align*}
    The Sobolev Embedding Theorem for $n=2$ only applies to $C_C^\infty$ functions in $\Rl^2$.  To get back to that result from this theorem, choose $n=2$, $p=\infty$, $k=1$, and $u \in C^\infty(\Omega)$ where $\Omega \Subset \Rl^2$.  Then we see that
    \begin{align*}
        \norm{u}_{\mathcal{C}^1(\Omega)} \leq C\norm{u}_{W^{1,\infty}(\Omega)}.
    \end{align*}
    Then since the continuous functions on compact sets are bounded, the $L^\infty$ norm is bounded by the $\mathcal{C}^1$ norm, and we get the result.
\end{flashcard}

\begin{flashcard}
    {What is the Sobolev Embedding Theorem for $kp > n$?}
    There is a constant $C = C(k,p,n)$ such that
    \begin{align*}
        \norm{u}_{\mathcal{C}^{k - \left\lfloor\frac{n}{p}\right\rfloor - 1,\gamma}(\Rl^n)} \leq C\norm{u}_{W^{k,p}(\Rl^n)}
    \end{align*}
    where
    \begin{equation*}
        \begin{aligned}
            \gamma = \begin{cases}
                \left\lfloor\frac{n}{p}\right\rfloor + 1 - \frac{n}{p} & \text{ if } \frac{n}{p} \not\in \mathbb{N} \\
                \text{any } \alpha \in \Rl \cap (0,1) & \text{ if } \frac{n}{p} \in \mathbb{N}
            \end{cases}
        \end{aligned}
    \end{equation*}
    This inequality relates a function's differentiability with its smoothness.  That is, if a function is in $W^{k,p}$, it is ``differentiable'' a certain number of times, i.e.~$W^{k,p}$ functions necessarily have an amount of ``smoothness''.
\end{flashcard}

\begin{flashcard}
    {What can we say about the differentiability of functions in $W^{1,p}_\text{loc}(\Omega)$ where $\Omega \subset \Rl^n$ and $n < p \leq \infty$?}
    If $u \in W^{1,p}_\text{loc}(\Omega)$, then $u$ is differentiable (a.e.) in $\Omega$ and its derivative is equal to the weak derivative (a.e.).
\end{flashcard}

\begin{flashcard}
    {What is the Gagliardo-Nirenberg-Sobolev Inequality for $k=1$?  Why do we need it?}
    For $1 \leq p < n$, let $p^* = \dfrac{np}{n - p}$, i.e.~$\displaystyle\frac{1}{p} - \frac{1}{p^*} = \frac{1}{n}$.  Then
    \begin{align*}
        \norm{u}_{L^{p^*}(\Rl^n)} \leq C(p,n)\norm{Du}_{L^p(\Rl^n)} \qquad \forall u \in W^{1,p}(\Rl^n)
    \end{align*}
    This inequality relates the decay of a function's derivative to its own decay.  If a function's derivative decays $L^p$ fast, then the function decays even faster, specifically, $L^{p^*}$.
\end{flashcard}

\begin{flashcard}
    {What is the Gagliardo-Nirenberg-Sobolev Inequality for $1\leq kp < n$?}
    Suppose $D^k u \in L^p(\Rl^n)$.  Then $u \in L^{\frac{np}{n-kp}}(\Rl^n)$ and
    \begin{align*}
        \norm{u}_{L^{\frac{np}{n-kp}}(\Rl^n)} \leq C\norm{D^ku}_{L^p(\Rl^n)}
    \end{align*}
    Also, if $\Omega \subset \Rl^n$ is open, bounded, has a $\mathcal{C}^1$ boundary, then
    \begin{align*}
        \norm{u}_{L^{\frac{np}{n-kp}}(\Omega)} \leq C\norm{u}_{W^{k,p}(\Omega)}
    \end{align*}
\end{flashcard}

\begin{flashcard}
    {What is Morrey's Inequality for $n < kp$?}
    Let $u \in W^{k,p}(\Rl^n)$.  Then
    \begin{align*}
        \norm{u}_{\mathcal{C}^{k - 1 - \left\lfloor\frac{n}{p}\right\rfloor, 1 + \left\lfloor\frac{n}{p}\right\rfloor - \frac{n}{p}}(\Rl^n)} \leq C\norm{u}_{W^{k,p}(\Rl^n)}
    \end{align*}
\end{flashcard}

\begin{flashcard}
    {What is the general Interpolation Inequality for $W^{k,p}$?}
    Let $n \in \mathbb{N}$ and $p,q,r,j,k,\ell$ satisfy the relations
    \begin{equation*}
        \begin{aligned}
            j \leq k < \ell \qquad \qquad p,q,r \geq 1 \qquad \qquad 0 < \alpha \leq 1 \\[.1cm]
            \frac{1}{p} - \frac{k}{n} = \alpha\qty(\frac{1}{q} - \frac{\ell}{n}) + \qty(1 - \alpha)\qty(\frac{1}{r} - \frac{j}{n}) \\[.1cm]
            \frac{1}{p} - \frac{k}{n} > \frac{1}{q} - \frac{\ell}{n} \geq \frac{1}{p} - \frac{k+1}{n}
        \end{aligned}
    \end{equation*}
    Then $\exists C = C(p,q,r,j,k,\ell)$ such that
    \begin{align*}
        \norm{u}_{W^{k,p}(\Rl^n)} \leq C\norm{u}_{W^{\ell,q}(\Rl^n)}^\alpha\norm{u}_{W^{j,r}(\Rl^n)}^{1-\alpha}.
    \end{align*}
    That is, if a function has a $q$-summable $\ell$\textsuperscript{th} derivative and an $r$-summable $j$\textsuperscript{th} derivative, then it has a $p$-summable $k$\textsuperscript{th} derivative.
\end{flashcard}

\end{document}

