\documentclass[avery5388,grid,frame]{flashcards}

\cardfrontstyle[\large\slshape]{headings}
\cardbackstyle{empty}

\usepackage{amssymb, amsmath, amsfonts}
\usepackage{mathtools}
\usepackage{physics}
\usepackage{enumerate}
\usepackage{array}

\newcommand{\E}{\varepsilon}
\newcommand{\ran}{\mathrm{ran}\,}
\newcommand{\ind}{\mathrm{ind}\,}
\newcommand{\Lip}{\mathrm{Lip}\,}
\newcommand{\supp}{\mathrm{supp}\,}
\newcommand{\sgn}[1]{\mathrm{sgn}\left[#1\right]}
\newcommand{\f}[3]{#1\ :\ #2 \rightarrow #3}
\def\Rl{\mathbb{R}}
\def\Cx{\mathbb{C}}
\def\hilb{\mathcal{H}}
\def\torus{\mathbb{T}}
\def\topo{\mathcal{T}}
\def\sopo{\mathcal{S}}

\begin{document}

\cardfrontfoot{Shkoller Analysis Chapter 2.1}


\begin{flashcard}
    {State the Divergence Theorem}
    Let $\Omega \subset \Rl^n$ be a Lipschitz domain and $w = (w_1, \dots, w_n) \in \mathcal{C}^1(\overline{\Omega})$ with outward pointing normal $N$.  Then
    \begin{align*}
        \int_\Omega \div w\ \dd x = \int_{\partial \Omega} w \cdot N \dd S
    \end{align*}
\end{flashcard}

\begin{flashcard}
    {State Green's First and Second Identities.}
    Let $\Omega$ be a smooth domain and let $u \in \mathcal{C}^2(\overline{\Omega})$ and $v \in \mathcal{C}^1(\overline{\Omega})$-funtions.  Then we have Green's First Identity:
    \begin{align*}
        \int_\Omega \grad v \cdot \grad u + v\laplacian u \dd x = \int_\Omega \div\qty(v\grad u)\dd x = \int_{\partial \Omega} v \frac{\partial u}{\partial N}\dd S
    \end{align*}
    Exchanging $u$ and $v$ in Green's First Identity and finding the difference gives Green's Second Identity:
    \begin{align*}
        \int_\Omega \qty(v \laplacian u - u\laplacian v)\dd x = \int_{\partial \Omega}\qty[v\frac{\partial u}{\partial N} - u \frac{\partial v}{\partial N}]\dd S
    \end{align*}
\end{flashcard}

\begin{flashcard}
    {What is a ``test function''?}
    Test functions are smooth functions with compact support, i.e.
    \begin{align*}
        \mathcal{C}_c^\infty = \left\{u \in \mathcal{C}^\infty\ :\ \supp u \subset \mathcal{V} \Subset \Omega\right\}.
    \end{align*}
\end{flashcard}

\begin{flashcard}
    {What is a weak derivative of an $L^1_{\text{loc}}$ function?}
    Let $u \in L^1_{\text{loc}}(\Omega)$.  Then $v^\alpha \in L^1_\text{loc}(\Omega)$ is called the $\alpha^\text{th}$ weak derivative of $u$, written $v^\alpha = D^\alpha u$, if
    \begin{align*}
        \int_\Omega u(x) D^\alpha \phi(x) \dd x = (-1)^{\abs{\alpha}}\int_\Omega  v^\alpha(x)\phi(x)\dd x \qquad \forall \phi \in \mathcal{C}_c^\infty(\Omega),
    \end{align*}
    where $\alpha \in \mathbb{N}^n$ is a multi-index with $\abs{\alpha} = \alpha_1 + \dots + \alpha_n$.
\end{flashcard}

\begin{flashcard}
    {Why does $f$ have a weak-derivative, but $g$ does not?
    \begin{equation*}
        \begin{aligned}
            f(x) = \begin{cases}
                x & \text{ if } x \in (0,1) \\
                1 & \text{ if } x \in (1,2)
            \end{cases}
        \end{aligned}
    \end{equation*}
    \begin{equation*}
        \begin{aligned}
            g(x) = \begin{cases}
                x & \text{ if } x \in (0,1) \\
                2 & \text{ if } x \in (1,2)
            \end{cases}
        \end{aligned}
    \end{equation*}}
    We can explicitly calculate the weak derivative (using integration by parts) of $f$.  For $g$, however, assuming a weak derivative exists results in a contradiction by exploiting the boundary terms in the integration by parts that don't cancel each other out.
\end{flashcard}

\begin{flashcard}
    {Define $W^{1,p}(\Omega)$ for $1 \leq p \leq \infty$.  Then define $W^{k,p}(\Omega)$ for $1 \leq p \leq \infty$ and $k \in \mathbb{N}$.  What is the norm in $W^{k,p}$?}
    \begin{align*}
        W^{1,p}(\Omega) = \left\{u \in L^p(\Omega)\ |\ \text{the weak derivative $u'$ of $u$ exists, and } u' \in L^p(\Omega)\right\}
    \end{align*}
    \begin{align*}
        W^{k,p}(\Omega) = \left\{u \in L^1_\text{loc}(\Omega)\ |\ D^\alpha u \text{ exists and is in } L^p(\Omega) \text{ for } \abs{\alpha} \leq k\right\}
    \end{align*}
    \begin{align*}
        \norm{u}_{W^{k,p}(\Omega)} = \qty(\sum_{\abs{\alpha}\leq k}\norm{D^\alpha u}_{L^p(\Omega)}^p)^{\frac{1}{p}}
    \end{align*}
    \begin{align*}
        \norm{u}_{W^{k,\infty}(\Omega)} = \sum_{\alpha \leq k} \norm{D^\alpha u}_{L^\infty(\Omega)}
    \end{align*}
\end{flashcard}

\begin{flashcard}
    {What is the ``simple version'' of the Sobolev Embedding Theorem?}
    Let $kp > 2$ and $u \in \mathcal{C}_c^\infty(\Rl^2)$.  Then
    \begin{align*}
        \norm{u}_{L^\infty(\Rl^2)} \leq C \norm{u}_{W^{k,p}(\Rl^2)}
    \end{align*}
    In English, $W^{k,p}$ functions are bounded in $\Rl^2$.
\end{flashcard}

\end{document}

