\documentclass[avery5388,grid,frame]{flashcards}

\cardfrontstyle[\large\slshape]{headings}
\cardbackstyle{empty}

\usepackage{amssymb, amsmath, amsfonts, amsthm}
\usepackage{mathtools}
\usepackage{physics}
\usepackage{enumerate}
\usepackage{array}
\usepackage{enumitem}
\usepackage{hyperref}


\newcommand{\N}{\mathbb{N}}
\newcommand{\Z}{\mathbb{Z}}
\newcommand{\Q}{\mathbb{Q}}
\newcommand{\R}{\mathbb{R}}
\newcommand{\rar}{\rightarrow}
\newcommand{\E}{\mathbb{E}}
\newcommand{\Oc}{\mathcal{O}}
\newcommand{\la}{\langle}
\newcommand{\ra}{\rangle}
%\newcommand{\alert}[1]{\textcolor{red}{#1}}
\newcommand{\Var}{\text{Var}}
\newcommand{\sigf}{\sigma-\text{field}}
\newcommand{\F}{\mathcal F}


\newtheorem*{theorem}{Theorem}
%\newtheorem{proof}[Proof]
\newtheorem*{example}{Example}
\newtheorem*{definition}{Definition}
\newtheorem*{properties}{Properties}
\newtheorem*{remark}{Remark}



\begin{document}
\cardfrontfoot{Bickel and Docksum}


\begin{flashcard}
    {What is Bayes risk? What is a Bayes rule?}
    \begin{definition}
        Given a loss function $l$ and a decision procedure $\delta$ and a set of probability distributions $P_\theta$ parameterized by $\theta$ with a distribution $P'$ on the $\theta$ the Bayes risk is
        $$r(\delta) = \E[l(\delta(X),\theta)]$$
        where the expectation is taken over $P'$ and $P_\theta$.
    \end{definition}

    \begin{definition}
        A Bayes rule is a decision procedure $\delta^*$ with
        $$r(\delta^*) = \min_\delta r(\delta)$$
    \end{definition}
\end{flashcard}


\begin{flashcard}
    {What is minimax risk?}
    \begin{definition}
        Given a loss function $l$ and a decision procedure $\delta$ and a set of probability distributions $P_\theta$, the max risk is
        $$R(\delta) = \sup_\theta \E[l(\delta(X),\theta)]$$
    \end{definition}

    \begin{definition}
        The minimax procedure is a procedure $\delta^*$ with
        $$R(\delta^*) = \min_\delta r(\delta)$$
    \end{definition}
\end{flashcard}


\begin{flashcard}
    {What is mean squared error and how does it decompose?}
    \begin{definition}
        The mean squared error is the risk under the quadratic loss function:
        $$R(\delta(X),\theta) = \E[(\delta(X) - \theta)^2] = \text{Bias}(\delta)^2 + \text{Var}(\delta)$$
    \end{definition}
\end{flashcard}


\begin{flashcard}
    {What are one parameter exponential families?}
    \begin{definition}
        Probability densities with the form
        $$p(x,\theta) = h(x) \exp \{ \eta(\theta)T(x) - B(\theta) \}$$
    \end{definition}

    \begin{example}
        Normal, Binomial, Poisson, exponential.
    \end{example}
\end{flashcard}


\begin{flashcard}
    {What is a risk set?}
    \begin{definition}
        The risk set is the set of risk vectors for every decision rule
        $$S = \{ (R(\theta,\delta))_{\theta \in \Theta} \mid \delta \in \mathcal D \}$$
        where $\mathcal D$ is the set of decision rules and $R(\theta,\delta)$ is the risk for the particular value of $\theta$.
    \end{definition}

    \begin{remark}
        The risk set is a finite set of points in $\R^{|\Theta|}$ if the action space for the decision rules is finite. If stochastic decisions are allowed, the risk set becomes the convex hull of this finite set. For different priors $\pi$, the risk for a stochastic strategy is $\la \pi, (R(\delta,\theta)) \ra$, hence the level sets of Bayes risk are $\la \pi, \cdot \ra = c$. The Bayes estimator is the solution to this linear programming problem over a convex hull and hence is either on a line of the boundary or on a corner. Hence, deterministic strategies do well enough for Bayes risk.
    \end{remark}

    \begin{remark}
        For minimax strategies, the problem, geometrically, is about finding the smallest square that fits below the convex hull.
    \end{remark}
\end{flashcard}


\begin{flashcard}
    {What is an admissible rule?}
    \begin{definition}
        A rule is admissible if it's not inadmissible. An inadmissible rule is one where there exists another rule that improves on it.
    \end{definition}
\end{flashcard}


\begin{flashcard}
    {What is the relationship between minimax rules and Bayes rules?}
    \begin{theorem}
        If $\Theta$ is finite and minimax procedures exist, they are Bayes procedures. (see pg 49 in Bickel for more references)
    \end{theorem}
\end{flashcard}


\begin{flashcard}
    {When are stochastic decisions an improvement on deterministic ones?}
    \begin{theorem}
        When $\Theta$ is not finite, typically Bayes procedures are not admissible. (see pg 49 in Bickel for more references)
    \end{theorem}
\end{flashcard}


\begin{flashcard}
    {}

\end{flashcard}


\begin{flashcard}
    {}

\end{flashcard}


\begin{flashcard}
    {}

\end{flashcard}


\begin{flashcard}
    {}

\end{flashcard}


\end{document}
